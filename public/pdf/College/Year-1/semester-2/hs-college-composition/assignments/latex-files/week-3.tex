\begin{essay}
  % My intro paragraph
  When I was younger, about $8$ - $9$ years old, I enjoyed watching shows like
  \textbf{Popular Mechanics} and \textbf{National Geographic}. Unfortunately, I
  regretted watching one of the \textbf{National Geographic} videos for almost
  $2$ years. The video talked about sinkholes, which terrorized me. Since I was
  a very optimistic, naive little kid who didn't think anything was wrong in the
  world, this took its toll on me. My bubble of a world shook. The world wasn't
  all flowers and roses anymore. I created many dark ideas about how the world
  may end. From fire and flames to floods and light beaming from the sky. From
  that day on, I imagined every day as the last day, spending countless hours
  reading about sinkholes. Every time I tried to sleep, I would hear the
  piercing sound coming from the sinkholes and visually see it as if I was
  falling into one. My family forbade me from reading any more books on the
  matter. Eventually, like all little kids, I grew out of this phobia.

  % Inline etymological definition of the term apocalypse
  The online etymological dictionary defines the term \textbf{Apocalypse} as
  \textit{a cataclysmic event}. There are other ways to interpret the term, which
  is what I'll be discussion with you.

  % Talk about the Quranic End of Times
  A smoke so thick you cannot see your hand, even if it's an inch from your face
  covers the entire atmosphere. A beast comes to talk to you while there are
  earthquakes shaking the entire world. Fires ravage from Yemen, which forces
  everybody to one location. This is how the Islamic religion depicts the end of
  times. I grew up with this story if how the end of times will commence. I
  didn't really take it too seriously, but eventually, I decided to learn more
  about it. It's narrated that after Prophet Muhammad lead the morning prayer
  (\textbf{fajir}), which is around $4$ AM, he got on his mimbar and he preached
  until the noon prayer (\textbf{duhr}), which is around $1$ PM. After they
  finished praying, he got back up and preached until the next prayer, and the
  next and the final prayer, which is around $11$ PM. He's describing the end of
  times. You can find dozens of volumes of books talking about this subject and
  that doesn't even cover half of everything. There are two chapters in the
  Quran which are dedicated to talking about this specific event. The first one,
  Al-Waqiah, which means \textbf{The Inevitable} and Al-Qiyamah, which means
  \textbf{The Resurrection}. Prophet Muhammad stated that the End of Days won't
  happen until you see the $10$ signs. The first sign is a thick smoke that will
  cover our entire atmosphere. After that, the Dajjal, or Anti-Christ will come
  while a beast comes to talk to the people. One of the most dang ours signs is
  when the sun rises and sets on opposite directions. The reason this is
  dangerous is because when this happens, Allah won't forgive anybody and if
  anybody wants to become Muslim after this event, it won't be accepted. Then,
  Jesus PBU descends to kill the Dajjal, which is right before the Gog and Magog
  arrival. The next sign are a group of $3$. $3$ earthquakes shake the entire
  earth. One in the East, one in the West, and one in the Arabian Peninsula. The
  final one is a fire from Yemen will start to gather all the people around the
  world to the place called Mahshar, which means the gather place. This is where
  the people will be resurrected for the day of judgment. Prophet Muhammad said
  if you see only one of these signs, be prepared to see the next ones very
  soon. While all of these signs are happening, the world will split into $2$
  categories. Muslims vs Christians. The Prophet Muhammad called this war the
  \textbf{Great War}. He also said that there will be no affliction that will be
  worse than this war. I'm not going to go in detail because I would end up
  writing an entire volume of books, but a key highlight is when the people are
  gathered in the Mahshar, a breeze will come down and whoever has an ounce of
  faith will die peacefully while the non-believers will die later on. After
  every body dies, Allah folds the entire Universe with one hand and the other
  $6$ universes with the other. After a while, he calls everybody to be judged
  for what they have done.

  The \textbf{Train to Busan} is an apocalyptic Korean zombie movie. The main plot
  of the movie is about a father and his daughter trying to make their way across
  \textbf{South Korea}, via a train, when, out of nowhere, a virus starts to
  spread among the passengers. The said virus kills the victim and makes them
  become a zombie. It focuses on different aspects, which other zombie movies may
  not. It criticizes society on many levels, one being, how humans are and how an
  event so colossal as an apocalypse can show their true skin. The characters in
  the movie make this movie distinct because they're very likable people since
  they aren't horrible to other people and selfless. Usually, we see one main
  character who survives by sheer luck or by sacrificing everybody to save
  himself. In this movie, the clever ones are the people standing in the end if
  they decide not to sacrifice themselves to save everybody else.

  Initially, I assumed that the term \textbf{Apocalypse} meant the end of the
  world. Although that definition is a worldwide accepted meaning, there are other
  definitions and interpretations. One interpretation may be \textit{an Apocalypse
  brings out the true color in people.} It will help you distinguish your friends
  from your enemies and vice versa. Also, an apocalypse may be the end of a
  chapter and the beginning of something new like how all the religious scriptures
  say. Even in the \textbf{Train to Busan}. The apocalypse may start something new
  like a discovery about something from a different world. That's also the main
  theme for the Quranic end of times, where everybody will turn against each
  other to blame for their sins, which reveals who they truly are. One thing
  that struct me the most was how my fear of sinkholes started to reveal to me
  about a world that I never knew, one that no religious text, or even
  apocalyptic movies/books can display. My fear of sinkholes gathered my family
  together to help me, even if it's not the best way possible. On the other
  hand, the Quranic text shows what will happen in the far out future while my
  fear displayed fears of everyday life. The Train to Busan displayed that
  really well for me because the virus just appeared out of nowhere, which is
  what I thought would happen with sinkholes. Overall, I'm glad that I went
  through with the troubles since it helped develop my personality and widen my
  view of the world.
\end{essay}
